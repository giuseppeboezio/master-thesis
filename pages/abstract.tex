% !TEX encoding = UTF-8
% !TEX TS-program = pdflatex
% !TEX root = ../tesi.tex

%**************************************************************
% Sommario
%**************************************************************
\cleardoublepage
\phantomsection
\pdfbookmark{Abstract}{Abstract}
\begingroup
\let\clearpage\relax
\let\cleardoublepage\relax
\let\cleardoublepage\relax

\chapter*{Abstract}

%\vfill
%
%\selectlanguage{english}
%\pdfbookmark{Abstract}{Abstract}
%\chapter*{Abstract}
%
%\selectlanguage{italian}
The ability to create hybrid systems that blend different paradigms has now become a requirement for complex AI systems usually made of more than a component. In this way, it is possible to exploit the advantages of each paradigm and exploit the potential of different approaches such as symbolic and non-symbolic approaches. In particular, symbolic approaches are often exploited for their efficiency, effectiveness and ability to manage large amounts of data, while symbolic approaches are exploited to ensure aspects related to explainability, fairness, and trustworthiness in general.\newline\newline
The thesis lies in this context, in particular in the design and development of symbolic technologies that can be easily integrated and interoperable with other AI technologies.
2P-Kt is a symbolic ecosystem developed for this purpose, it provides a logic-programming (LP) engine which can be easily extended and customized to deal with specific needs.\newline\newline
The aim of this thesis is to extend 2P-Kt to support constraint logic programming (CLP) as one of the main paradigms for solving highly combinatorial problems given a declarative problem description and a general constraint-propagation engine.
A real case study concerning school timetabling is described to show a practical usage of the CLP(FD) library implemented.\newline\newline
Since CLP represents only a particular scenario for extending LP to domain-specific scenarios, in this thesis we present also a more general framework: Labelled Prolog, extending LP with labelled terms and in particular labelled variables.
The designed framework shows how it is possible to frame all variations and extensions of logic programming under a single language reducing the huge amount of existing languages and libraries and focusing more on how to manage different domain needs using labels which can be associated with every kind of term.\newline\newline
Mapping of CLP into Labeled Prolog is also discussed as well as the benefits of the provided approach.

\endgroup

\vfill

