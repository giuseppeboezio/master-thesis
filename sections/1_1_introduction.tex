Nowadays there are a lot of hybrid systems which exploit on different paradigms to deal with complex A.I. problems.
An example could be represented by systems which use both symbolic and sub-symbolic approaches. Different solutions have been proposed during
years; for example in \cite{TOWELL1994119} it is described how to reformulate rules in such a way that they can be directly used in an
artificial neural network or in \cite{https://doi.org/10.48550/arxiv.2209.12618} it is described how to use together neural networks and formal logic to be able to rely on
strenghts of both approaches in different fields such as healthcare, image processing, business management, information retrieval and many other fields.\newline\newline
2P-Kt is an extensible ecosystem written in Kotlin which can be easily used by other applications to perform tasks using logic programming with Prolog programming language.
Logic programming has developed during years different variations and extensions according to the needs of a specific domain. One flavor is constraint logic programming (CLP) which
consists of applying techniques to solve combinatorial problems using an approach based on logic programming instead of using
other kind of techniques based on other paradigms. These techniques are together called Constraint Programming.
The first contribution of this thesis is to endow 2P-Kt ecosystem with a set of libraries of CLP which allow users to be able to use this interpreter to solve specific
problems related to combinatorial domain. To do this the Prolog predicates provided by this new group of libraries have been realized as close as
possible to existing CLP libraries to make easier people already familiar with CLP to switch to these libraries without spending time to learn new
predicates and their usages. Libraries have been implemented adapting an existing CP library called Choco with some 2P-Kt mechanisms which allow
to mask the underneath CP implementation providing common predicates supported by a very famous and used CLP library of SWI Prolog interpreter called
CLP(X).\newline\newline
A real case study has been developed to show how these libraries could be used by other systems to integrate different kind of functionalities. In the current case study the library concerning
integer variables (CLP(FD)) has been used to model constraints among professors' lessons in an italian high school. The library was very useful because it was possible to integrate
Constraint Logic Programming in a Multi-agent context concering different kind of languages and approaches allowing the realization of a complex hybrid system as mentioned earlier.\newline\newline
This particular extension of the 2P-Kt ecosystem has allowed to reason on how it would be unfeasible to extend the system for every kind of
specific application domain and needs, therefore, exploiting on the concept of Labelled Variables, a new framework has been
developed which extends the label concepts to all Prolog terms. This approach allows to reduce an huge amount of new Prolog variations and extensions
focusing on how to adapt specif domain issues to the label concept.\newline\newline
A possible approach to CLP with labels has been proposed to show how the labels could be effectively used to support different aspects of
constraint programming with labels provided by this new framework.

