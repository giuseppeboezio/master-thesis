\section{Implementation}\label{sec:Implementation}

The aforementioned libraries have been implemented in 2P-Kt using the following interfaces and classes:
\begin{itemize}
    \item \textbf{PrimitiveWrapper}: it is an abstract class which allows to define a predicate which is not unified as usual but it executes some code producing some
    side effect on the actual substitution. This class reifies the generator concept described in \cite{10.1007/978-3-030-75775-5_27}
    where the solver can be seen as a stream consumer allowing it to get a stream of solutions interacting with a primitive mechanism (the generator)
    which can be seen by the solver as an ordinary build-in predicate denoted by its own \textit{signature} and \textit{arity}. The PrimitiveWrapper has been used to define most of predicates contained in the libraries.
    \item \textbf{RuleWrapper}: it is an abstract class which allows to define a Prolog \textit{Rule}. This class is useful to avoid
    repeated code because it allows to define some predicates in terms of other existing ones.
    \item \textbf{Library}: it is an interface which has been implemented to group together different predicates implementations which
    can be either \textit{PrimitiveWrapper} or \textit{RuleWrapper}
    \item \textbf{durable CustomData}: it is a map from String to Any which allows to store data during the resolution process. It has been used to store the
    model containing all variables and constraints of the problem
    \item \textbf{DefaultTermVisitor}: this abstract class is used to realize the visitor pattern \cite{gamma1994design} and it has been
    extensively used for different tasks such as evaluation of expressions, generation of arithmetic expressions and all other tasks related to perform different operation wrt the type of the term
\end{itemize}


The actual implementation of all constraints have been realized exploiting on the Choco Library \cite{Prud'homme2022}. This library has been chosen for different reasons:
\begin{itemize}
    \item it is written in Java and can be easily used with Kotlin language
    \item wrt other similar libraries (e.g. OR Tools \cite{ortools} or JaCoP \cite{Kuchcinski2013JaCoPJ}) it simplifies the composition of expressions for the creation of new variables or constraints
    \item well documented and with an active community to get support in case of doubts or problems
\end{itemize}

Choco Library cannot be directly mapped with SWI Prolog CLP(X) libraries
because these these libraries belong to a different paradigms. For this reason the mapping was not immediate but
sometimes different approaches have been used to solve these issues.\newline\newline
Following the main mapping choices divided by Library

\subsection{Constraint Logic Programming over Finite Domains}\label{map_clpfd}







