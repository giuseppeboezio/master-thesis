Nowadays \textit{pervasive systems} are a challenge which requires suitable models and technologies to support
\textit{distributed situated intelligence}. Logic Programming (LP) can be used as the core of such models and technologies thanks to its declarative interpretation and inferential capabilities
but, it is unsuitable to capture different domain-specific computational models. For this reason, logic programming needs
to be extended delegating other aspects, such as situated computations, to other languages, or, to other levels of computation \cite{10.3233/FI-2018-1695}.\newline
In the following chapter two different approaches will be presented: \textit{Labelled Variables in Logic Programming} (LVLP) and
\textit{Labelled Terms in Logic Programming} (LTLP).\newline\newline
\textit{Labelled Variables in Logic Programming} consists of enabling different computational models adding information (called \textit{labels}) to variables
whereas \textit{Labelled Terms in Logic Programming} extends the label concept to all possible kind of terms in a logic program.\newline\newline
Both LP extensions have been implemented for the Prolog language. This first one has been described in \cite{10.3233/FI-2018-1695} whereas the second one will be explained more in details in further section \ref{sec:implementation_labelled}.