\section{Model}\label{sec:model}

\subsection{Labelled Variables}\label{subsec:labelled_variables}

Let \textit{C} be the set of constants, with c1, c2 $\in$ \textit{C} being two generic constants. Let \textit{V} be the set of
variables, with v1, v2 $\in$ \textit{V} being two generic variables. Let \textit{F} be the set of function symbols, with
f1, f2 $\in$ \textit{F} being two generic function symbols; each f $\in$ F is associated to an arity ar(f) > 0,
stating the number of function arguments. Let \textit{T} be the set of terms, with t1, t2 $\in$ \textit{T} being two
generic terms. Terms can be either simple, a constant (e.g., c1) and a variable (e.g., v2) are both
simple terms, or compound, a functor of arity n applied to n terms (e.g., f1(c2, v1)) is a compound term. A term is said ground if it does not contain variables
Let \textit{H} denote the set of ground terms, also known as the Herbrand universe. A model for Labelled Variables in Logic Programming (LVLP) is defined as a triple ⟨\textit{B}, $f_L$, $f_C$⟩,
where
\begin{itemize}
    \item \textit{B} = \{$\beta_1, . . . , \beta_n$\} is the set of basic labels, the basic entities of the domain of labels
    \item \textit{L} $\subseteq \mathcal{P}(\textit{B}) $ is the set of labels, where each label \textit{l} $\in$ \textit{L} is a subset of \textit{B}
    \item $f_L : \textit{L} \times \textit{L} \rightarrow \textit{L}$ is the (label-)combining function computing a new label from two given ones
    \item $f_C : \textit{H} \times \textit{L} \rightarrow {true, false}$ is the compatibility function, assessing the compatibility between
    a ground term and a label when interpreted in the domain of labels
    \item a labelled variable is a pair ⟨v, l⟩ associating label l $\in$ L to variable v $\in$ V
    \item a \textit{labelling} is a set of \textit{labelled variables}
\end{itemize}

An LVLP program is a collection of LVLP rules of the form
\[ Head :- Labelling, Body. \]
to be read as “Head if Body given Labelling”. There, Head is an atomic formula, Labelling is the
list of labelled variables in the clause, and Body is a list of atomic formulas.
An atomic formula has the form $p(t_1,...,t_n)$ where p is a
predicate symbol and $t_i$ are terms. Atom $p(t_1,...,t_n)$ is said ground if $t_1,..., t_n$ are ground.\newline\newline
The \textit{unification} between terms happens as described in the following table:


The only case to be added to the standard unification table is represented by labelled variables.
There, given two generic LVLP terms, the unification result is represented by the extended tuple
\[ (true/false, \theta, l) \]
where true/false represents the existence of an answer, θ is the most general unifier \(mgu\), and l is
the new label associated to the unified variables defined by the (label-)combining function $f_L$
It is important to define a concept which allows to check whether a label assigned to a variable is compatible with its substitution or not.
To do this, the following \textit{compatibility function} $f_C$ is defined:
\[ f_C : H \times L \rightarrow \{true, false\}\]
In particular, given a a ground term t $\in$ H and label l $\in$ L:
\begin{equation}
    f_C(t,l) = 
    \begin{cases*}
        true & if there exists at least one element of the domain of
        labels which the interpretations of t and l both refer to \\
        false & otherwise
    \end{cases*}
\end{equation}

\subsection{Labelled Terms}\label{subsec:labelled_terms}
Labelled Terms can be seen as a generalization of the previous approach where it is possible to apply a label not only to a variable
but to each kind of term. The main idea of this approach consists of using two functions:
\begin{itemize}
    \item \textbf{shouldUnify(term1, labels1, term2, labels2)}: this function is used to check whether, according to \textit{labels1} and \textit{labels2},
    it is possible to unify \textit{term1} and \textit{term2}. This function is very similar to $f_L$ used in
    the unification process to allow the substitution of a variable when $f_L(labels1,labels2) \neq \emptyset$
    \item \textbf{merge(term1, labels1, term2, labels2)}: this function is used to generate new labels after unifying terms \textit{term1} and \textit{term2}.
    It has the same meaning of $f_L(labels1,labels2)$ but it is applied to every kind of term
\end{itemize}
Unification between terms happens as described by the following table:

A specific issue of this generalization of labels to all kind of terms consists of considering the compatibility between the labels of a predicate and
the labels of its arguments.\newline
This is reasonable because a predicate expresses a relationship among objects; therefore the information (labels) associated to these objects
must be meaningful with respect to the information associated to their relation.\newline
For this reason the function \textbf{stillValid(struct)} has been added to the resolution process to check the compatibility
of the aforementioned labels.

