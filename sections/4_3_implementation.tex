\section{Implementation}\label{sec:implementation}

Implementation has been realized exploiting on 2P-Kt framework (previously described in \ref{subsec:2p_kt}).
Different choices have been made trying to extend as much as possible existing components provided by this framework.
Choices will be described according to the aspects they affect.

\subsection{Labels}\label{subsec:labels}

\textbf{Labels} have been attached to each term using a map from \textit{String} to \textit{Any} called tags. \textit{Tags}
are a machanism which can be realized by all classes which implements \textbf{Taggable} interface. \textbf{Term} implements
this interface therefore this map is used to associate to a tag (corresponding to labels concept) to a set of Label where a Label can be seen as a generic class
which implements the \textbf{Label} interface. Some examples have been provided where \textit{labels} are represented as strings.\newline
Adding \textit{labels} to a term required also to change how the new term is formatted. For this reason \textbf{LabelAwareTermFormatter} has been created
to properly format labelled terms. Each label is written preceeded by '@' symbol and all labels of a term are surrounded
bu angular brackets.\newline
For example the struct \textbf{f(a,B)} having \textbf{a} associated with labels {x,y} is formatted as:
\[f(a<@x,@y>,B)\]

\subsection{Unificator}\label{subsec:unificator}
\textbf{Unificator} interface has been extended to support the functions described in \ref{subsec:labelled_terms}.
\textbf{AbstractUnificator} class has been modified in order to allow the customization of the final substitution that in the current scenario
has been implemented to keep all labels associated to the different terms compared during the generation of the most general unifier (m.g.u.).\newline
The class \textbf{AbstractLabelledUnificator} has been created to allow the user to define own callback functions (shouldUnify and merge) in order
to deal with the specific scenario which requires specific kinds of labels and unification rules.

\subsection{Solver}\label{subsec:solver}
