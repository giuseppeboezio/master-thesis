\textit{Labelled Prolog} is a monotonic extension of Prolog which allows to deal with specific domains
without the need to use specific languages or libraries to deal with them.\newline
There are different variants and extensions of logic programming which include:
\begin{itemize}
    \item abductive logic programming \cite{10.1093/logcom/2.6.719}
    \item metalogic programming  \cite{10.5555/94469}
    \item constraint logic programming \cite{JAFFAR1994503}
    \item concurrent logic programming \cite{10.1145/800223.806776}
    \item probabilistic logic programming \cite{NG1992150}
\end{itemize}
With \textbf{Labelled Prolog} discussed in the previous chapter, all these extensions and
variants could be framed in a single environment allowing to reduce the amount of languages to a single one (Labelled Prolog) and
focusing more on how to adapt labels and unification mechanisms to the specific domain needs.\newline\newline
In the current chapter is shown how to adapt CLP to the labels mechanism using the following example.\newline
Let's assume to have the following \textit{Knowledge Base}:\newline
\begin{lstlisting}[label=program1, frame=none, mathescape, language=tuProlog, caption={}]
    problem(X<@1,@100>,Y<@1,@100>)<@>> :- [X,Y] ins 1..100, X #> Y.
    problem(X<@1,@100>,Y<@1,@100>)<@<> :- [X,Y] ins 1..100, X #< Y.
    problem(X<@1,@100>,Y<@1,@100>)<@>=> :- [X,Y] ins 1..100, X #>= Y.
\end{lstlisting}
A possible \textit{Goal} could be:
\begin{lstlisting}[label=program1, frame=none, mathescape, language=tuProlog, caption={}]
    ?-problem(X<@2,@8>,Y<@7,@12>)<@>>
\end{lstlisting}
With the current example the idea would be to state that we want to solve a problem whose variables have a domain
which is described by their labels and whose constraints are described by the labels of predicate \textit{problem/2}.\newline
In this way we are able to express a CLP problem in a much more expressive way wrt a way which is strictly related to a specific
language or library. The unification process will check, according to the constraint described as label, which is the rule
to execute. Domain of variables are combined using intersection and in this case the \textit{stillValid} callback
checks, in addition to the equality of the constraint, whether assignments of variables are between the first and the second label
of the variables (they correspond to the lower and upper bounds).\newline
In this example the body of the rules contains a specific domain range because with the current implementation it is not possible
to use labels as arguments of a predicate but only to attach them to terms. Therefore, a default domain range is
provided for both variables.
