Different results have been achieved with the current thesis and they can be described according to the area they affect.\newline\newline
From the Constraint Logic Programming perspective we have endowed the 2P-Kt interpreter with CLP functionalities. This means that all users of this interpreter
can be able now to write programs involving Constraint Programming using a familiar group of predicates already available which
make easier to switch to these CLP libraries. An interesting evolution of CLP in 2P-Kt could be to allow Constraint Logic Programming to be supported
with different Constraint Programming libraries. Currently only Choco Library is supported but it could be useful to develop a framework with a number of predefined predicates
which can be plugged to a specific library implementation in a similar way supported by Minizinc \cite{10.1007/978-3-540-74970-7_38}.\newline\newline
From a theoretical perspective CLP has allowed to reason more about specific needs of a domain and to generalize the Labelled Variables approach
to every possible kind of term changing the approach to a generic problem which involves logic programming (in particular Prolog) delegating the specific requirements
and needs of a domain to the label mechanism. This approach points out be successful because it requires the user to specify only
functions which take into account the specific scenario without caring about the overall resolution strategy.\newline
This could be seen as a starting point of a research inolving how to adapt existing variations
and extensions of Logic Programming to this new mechanism trying to understand also mathematical properties which can be entailed by it.\newline
In the currrent project we have dealt with more the ideas and implementation of this new mechanism instead of describing more in the details
the semantic of the language. Therefore, it could be interesting also to delve this aspect from a mathematical point of view in such a way to have a more
rigorous definition of this framework.\newline\newline
Prolog has been the main programming language used to describe this new approach but reasonably it could be possible to use it for a generic language which,
similarly to Prolog, describes the problem with a variant of First-Order Logic (FOL).